\documentclass[a4paper, 11pt]{article}
\usepackage{graphicx}
\usepackage{amsmath}
\usepackage[pdftex]{hyperref}

% Lengths and indenting
\setlength{\textwidth}{16.5cm}
\setlength{\marginparwidth}{1.5cm}
\setlength{\parindent}{0cm}
\setlength{\parskip}{0.15cm}
\setlength{\textheight}{22cm}
\setlength{\oddsidemargin}{0cm}
\setlength{\evensidemargin}{\oddsidemargin}
\setlength{\topmargin}{0cm}
\setlength{\headheight}{0cm}
\setlength{\headsep}{0cm}

\renewcommand{\familydefault}{\sfdefault}

\title{Data Mining: Learning from Large Data Sets - Fall Semester 2015}
\author{mwurm@student.ethz.ch\\ merkim@student.ethz.ch\\ lwoodtli@student.ethz.ch\\}
\date{\today}

\begin{document}
\maketitle

\section*{Approximate near-duplicate search using Locality Sensitive Hashing} 
Briefly describe the steps used to produce the solution. Feel
free to add plots or screenshots if you think it's necessary. The
report should contain a maximum of 2 pages.

\subsection{Problem Description}

\subsection{Aproach of the Team}
The team decided to develop the algorithms indiendantly in a first step.
Afterwards the implementations were compared. So the details could be
discussed and errors were eliminated.
At the end the team decided which implementation should be improved for
submission.

\subsection{Environment}
The runnig environment with a MapReduce framework was provided. So the
Mapper and the Reducer were developed indipendantely and then submitted to
the running environment.

\subsubsection{Map}

The Mapper does the main part of the work. It analyzes the shingles of the
videos and submits potential douplicates.

For that it uses a Locality Sensitive Hashing aproach.

... Min Hashing ... LSH (bands) ....

Resulting potential video pairs are sent to the Reducer. To improve the precision
of the Reducer the shingles of the videos are also provided. So the reducer can
do a real comparisons of the videos.

But it's still important to have an optimal Mapper step. Missed potential douplicates
(false negatives) and wrong potental douplicates (false positives) can ... the result.

false negatives: They are never compared by the reducer and are missed totally.

false positiver: The reducer needs to compare to much potential douplicates and is not
efficient anymore.




\subsubsection{Reduce}

Comparison of potential douplicates for 90%???...


\subsection{Improvement}

\subsubsection{Tuning of Parameters}

\subsubsection{Comparing of Candiate Pairs}

\subsection{Possible Improvement}

- Final step after all reducers for eliminating douplicates

\subsection{Conclusion}


\end{document} 
